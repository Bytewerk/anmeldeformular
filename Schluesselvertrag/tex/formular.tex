\begin{Form}
	{\centering
		\large
		\textbf{Vereinbarung zwischen}\\
	}
	\vspace{1em}
	\begin{tabu}{@{}XcX@{}}
		\bingoevNameLang & und & \ShVorname{} \ShName \\
		\bingoevAddrStrasse && \ShAddrStrasse \\
		\bingoevAddrStadt && \ShAddrStadt \\
											 && Login: \ShLogin \\
		im Folgenden \textit{\bingoevName} genannt && im Folgenden Schlüsselhalter (\textit{SH}) genannt.
	\end{tabu}
	
	\textbf{Schlüsseldaten siehe Rückseite.}\\
	
	Der Schlüssel ist ein persönlicher Schlüssel des SH. Er darf ausschließlich nach Rücksprache und mit schriftlicher Genehmigung durch den Vorstand an Dritte weitergegeben werden.\\
	
	Für den Schlüssel ist vom SH eine Kaution von \textbf{\ShKaution}\, Euro zu hinterlegen. Die Kaution wird auf einem Sparbuch verzinslich angelegt. Nach Rückgabe des Schlüssels erhält der SH die Kaution inkl. Zinsen zurück.\\
	
	Bei Schlüsselverlust haftet der SH persönlich und hat die Kosten für den Austausch der kompletten Schließanlage zu tragen. Dem SH wird der Abschluss einer entsprechenden Versicherung empfohlen.
	
	Der SH verpflichtet sich bei Nutzung der Räumlichkeiten:
	\begin{itemize}
		\setlength{\itemsep}{-5pt}
		\item Türen verschließen
		\item Heizung zudrehen, Fenster schließen, Licht, Elektrogeräte und PCs auszuschalten
		\item Meldung von Schäden oder Mängeln an die Geschäftsstelle
		\item Essen und trinken ist in den Schulungsräumen nicht gestattet
		\item Das Rauchen/Dampfen ist im gesamten Vereinsheim nicht gestattet
		\item Der SH sorgt für Ordnung und Sauberkeit sowie für den sorgfältigen Umgang mit allen Vereinseinrichtungen.
		\item Als Repräsentant des Vereins sorgt der SH dafür, dass andere Besucher die Regeln (insbesondere die Hausordnung) einhalten und sorgfältig mit den Einrichtungen des Vereins umgehen.
		\item Die Hausordnung in der jeweils gültigen Version wird vom SH anerkannt und eingehalten.
	\end{itemize}
	Bei Verstoß gegen die o.G. Regelung haftet der SH für dem Verein entstandene Schäden und muss den Schlüssel zurückgeben. Die Kaution wird in diesem Fall vom Verein einbehalten.\\
	
	Die oben genannten Bedingungen wurden gelesen und anerkannt\\
	
	\begin{tabu}{XXX[2]}
		Ort & Datum & Unterschrift \\
		\TextField[name=SIGN_ORT]{} & \TextField[name=SIGN_DATUM]{} & \TextField[name=SIGN]{} \\
	\end{tabu}
	
	
\end{Form}
