\documentclass[DIV=12,BCOR=1.5cm]{scrartcl}
\usepackage[ngerman]{babel}
\usepackage[T1]{fontenc}
\usepackage[utf8]{inputenc}
\usepackage[table]{xcolor}
% Prevent an option clash
\usepackage[letterhead,impressum]{bingoev}
\usepackage{tabu}
\usepackage{framed}
\usepackage{eurosym}
\usepackage{multicol}

\usepackage{hyperref}
\hypersetup{colorlinks,linkcolor=blue}

\renewcommand{\DefaultOptionsofCheckBox}{print,bordercolor=black}
\renewcommand{\DefaultOptionsofText}{print,bordercolor=black,borderstyle=U}
\renewcommand{\DefaultOptionsofComboBox}{print,sort,bordercolor=black,borderstyle=U}
\setlength{\parindent}{0pt}
\newcommand{\zuwendungsart}{Geldzuwendug}
\newcommand{\Titel}{Frau}
\newcommand{\Name}{Vor- und Nachname}
\newcommand{\Strasse}{Adresse mit nummer}
\newcommand{\Ort}{Ingolstadt}
\newcommand{\PLZ}{84049}

\newcommand{\BetragEuro}{21,00}
\newcommand{\BetragWort}{Einundzwanzig}
\newcommand{\ZuwendungDatum}{01.01.2016}
\newcommand{\Verwenugszweck}{Spende zweckgebunden}

\title{Bestätigung}
\date{}

\begin{document}
	\makeatletter
	\vskip1em
	{\usekomafont{title}\huge\@title}
	\vskip2em
	\makeatother
	über Zuwendungen im Sinne des § 10 b des Einkommensteuergesetzes an eine der in § 5 Abs. 1 Nr. 9 des Körperschafsteuergesetzes bezeichneten Körperschaften, Personalvereinigungen oder Vermögensmassen.\\
	\par\bigskip
	\begin{tabu}{p{7cm}|p{7cm}}
Art der Zuwendung & \zuwendungsart \\
\tabucline[1pt]{-}
 & \\
Name und Anschrift des Zuwendenden & \Titel \ \Name \\
 &  \Strasse \\
 &  \PLZ \ \Ort \\
	\end{tabu} \\
\par\bigskip
Betrag der Zuwendung: \\
\par\medskip
	\begin{tabu}{p{4cm}p{4cm}p{4cm}}
	Betrag in Ziffern & Betrag in Worten  & Tag der Zuwendung \\
	\tabucline[1pt]{-}
	 & & \\
	 Euro \BetragEuro & \BetragWort & \ZuwendungDatum \\
	\end{tabu}
\par\bigskip\bigskip
Verwendungszweck: \ \textbf{\Verwenugszweck}
\par\bigskip
Wir sind wegen Förderung der Erziehung, Volks- und Berufsbildung einschließlich der Studentenhilfe nach dem letzten uns zugegangenen Freistellungsbescheid des Finanzamtes Ingolstadt, Steuernummer 124/107/30497 vom 02.12.2015 für die Jahre 2012 bis 2014 nach § 5 Abs. 1 Nr. 9 des Körperschafsteuergesetzes von der Körperschafsteuer befreit und als gemeinnützig anerkannt.
\par\bigskip
Ingolstadt, \today 
\par\bigskip\bigskip\bigskip\bigskip\bigskip

\begin{tabu}{p{10cm}}
	\\
	\tabucline[1pt]{-}	
	Stempel und Unterschrift des Zuwendungsempfängers \\
\end{tabu}
\par\bigskip\bigskip\bigskip\bigskip
\begin{footnotesize}
Hinweis:\\
Wer vorsätzlich oder grob fahrlässig eine unrichtige Zuwendungsbescheinigung erstellt oder wer veranlasst, dass Zuwendungen nicht zu den in der Zuwendungsbescheinigung angegebenen steuerbegünstigten Zwecken verwendet werden, haftet für die Steuer, die dem Fiskus durch einen etwaigen Abzug der Zuwendung beim Zugewendeten entgeht (§10b Abs. 4 EStG, § 9 Abs. 3 KStG, § 9 Nr. 5 ewStG).
Diese Bestätigung wird nicht als Nachweis für die steuerliche Berücksichtigung der Zuwendung anerkannt, wenn das Datum des Freistellungsbescheides länger als 5 Jahre bzw. das Datum der vorläufigen Bescheinigung länger als 3 Jahre seit Ausstellung der Bestätigung zurückliegt (BMF vom 15.12.1994 - BStBI I S. 884).
\end{footnotesize}	
\end{document}