\section{Nutzungsbedingungen}
des Bürgernetz Ingolstadt, bingo e.V. für die Nutzung der vereinseigenen Kommunikationsmöglichkeiten.

\begin{multicols}{2}

\paragraph{Account/Bürgernetz}
\subparagraph{} Account i.S.d. Nutzungsbedingungen ist die Benutzerkennung – der Account-
Name und das zugehörige Passwort – sowie die damit verbundene
Genehmigung, die Kommunikationsmöglichkeiten des Bürgernetz Ingolstadt
zu benutzen.
\subparagraph{} Bürgernetz i.S.d. Nutzungsbedingungen ist die Gesamtheit der Programme
und Datenbestände, die bingo e.V. betreut, sowie die verwendete
Datenverarbeitungstechnik.

\paragraph{Datenschutz}
\subparagraph{} Der Verein stellt sicher, dass persönliche Daten nur für solche Zwecke
verwendet werden, die für die Vereinszwecke erforderlich sind.
Ausgenommen hiervon ist die Zusammenarbeit mit Strafverfolgungs- und
Ordnungsbehörden, zu denen der Verein verpflichtet ist,auch wenn ihm eine
Überprüfung der Rechtmäßigkeit solcher Ermittlungen im Einzelfall nicht
möglich ist.
\subparagraph{} Das Mitglied ist verpflichtet, dem Verein richtige und vollständige Angaben
über seine Person, Anschrift, Telefon- und Bankverbindung zur Verfügung zu
stellen und sie auf dem neuesten Stand zu halten.

\paragraph{Leistungsbeschreibung}
\subparagraph{} Grundsätzlich erhält jeder Nutzer einen Account und eine Email-Adresse.
\subparagraph{} Die Nutzungsmöglichkeiten orientieren sich an der Leistungsfähigkeit von bingo e.V.
\subparagraph{} Der Nutzer hat keinen Anspruch auf Zugang zum System, auf einwandfreien
Betrieb und das Angebot von Diensten.
\subparagraph{} Für das Einstellen von Informationen behält sich bingo e.V. eine Erhebung
von Unkostenbeiträgen vor.
\subparagraph{} Jeder Nutzer verpflichtet sich, mindestens monatlich seine E-Mail zu
bearbeiten. Erfolgt über einen Zeitraum von mehr als zwei Monaten kein
Zugriff auf E-Mail, erlischt die Nutzungsberechtigung.

\paragraph{Haftungsausschluss}
\subparagraph{} bingo e.V., seine Organe und Erfüllungs- und Verrichtungsgehilfen schließen
jegliche Haftung (aus vertraglicher und gesetzlicher Grundlage, insbesondere
positiver Forderungsverletzung und unerlaubter Handlung) für Schäden aus,
die dem Nutzer durch die Nutzung des Bürgernetz oder bingo e.V., seinen
Organen und Erfüllungs- und Verrichtungsgehilfen entstehen, sofern sie nicht
grob fahrlässig oder vorsätzlich herbeigeführt werden.
\subparagraph{} Für Schäden, die daraus entstehen, dass das Bürgernetz nicht oder nur
eingeschränkt nutzbar ist, übernimmt bingo e.V., seine Organe und
Erfüllungs- und Verrichtungsgehilfen weder gesetzliche noch vertragliche
Haftung.
\subparagraph{} Die Anbieter, ihre Organe und Erfüllungs- und Verrichtungsgehilfen haften
nicht für die über das Bürgernetz übermittelten Informationen und Daten
sowie deren Folgen, und zwar weder für Richtigkeit noch Vollständigkeit,
noch dass sie frei von Rechten Dritter sind oder der Sender rechtmäßig
handelt, indem er die Daten zugänglich macht, anbietet oder übersendet.

\paragraph{Nutzungsberechtigte}
\subparagraph{} Nutzungsberechtigt sind ausschließlich registrierte Personen, die sich mit den
vorliegenden Bestimmungen einverstanden erklären und sie befolgen.
\subparagraph{} Bei Minderjährigen ist die Einverständniserklärung der
Erziehungsberechtigten nachzuweisen.
\subparagraph{} Der Account wird widerruflich erteilt. Er kann jederzeit entzogen werden,
wenn gegen diese Nutzungsbestimmungen, gegen die Bestimmungen des
Bürgernetzverband e.V. oder gesetzliche Vorschriften verstoßen wurde.

\paragraph{Minderjährige}
\subparagraph{} Da es naturgemäß nicht auszuschließen ist, dass im Internet (wenn auch in
geringem Maße) auch jugendgefährdende und illegale Inhalte verbreitet
werden, sind die Erziehungsberechtigten verpflichtet, die Nutzung der
Accounts ihrer Kinder zu kontrollieren.

\paragraph{Überlassung des Accounts}
\subparagraph{} Die Überlassung des jeweiligen Accounts an Dritte ist aus Sicherheitsgründen
untersagt. Eine Überlassung des Accounts kann nicht nur zum Missbrauch
der Datenbestände des jeweiligen Benutzers, sondern auch des gesamten
Systems führen.

\paragraph{Protokoll}
\subparagraph{} bingo e.V. behält sich vor, Nutzungsverhalten und Systemzugriffe zu
protokollieren, um einen Missbrauch des Systems zu verhindern.
\paragraph{Missbräuchliche Nutzung}
\subparagraph{} Missbräuchliche Nutzung hat den Entzug des Accounts und gegebenenfalls
strafrechtliche Konsequenzen zur Folge.
\subparagraph{} Die Nutzung des Bürgernetzes ist vor allem dann missbräuchlich,, wenn das
Verhalten des Nutzers gegen die Nutzungsbestimmungen oder einschlägige
gesetzliche Schutzbestimmungen (z.B. Strafgesetz, Jugendschutzgesetz,
Datenschutzgesetz) verstößt.
\subparagraph{} Die Nutzung des Bürgernetzes ist missbräuchlich, wenn sie dazu dient,
illegale Handlungen damit zu begehen, zu fördern oder zu solchen
aufzufordern (z.B. Urheberrecht, Lizenzrecht, Persönlichkeitsrecht, Recht auf
informationelle Selbstbestimmung, Datenausspähung,
Computermanipulation, Computersabotage, Computerbetrug u.v.a).
\subparagraph{} Wer unberechtigt auf fremde Daten und Programme zugreift oder sie
verfälscht oder vernichtet, oder das System (sowohl das Bürgernetz als auch
das gesamte Internet mit all seinen angeschlossenen Computer) absichtlich
behindert oder beeinträchtigt, macht sich strafbar.

\paragraph{Informationsangebot im Bürgernetz}
\subparagraph{} \label{subpar:pruefunggenehmigung} Für Daten und Informationen, die in das Bürgernetz eingestellt werden, behält
sich bingo e.V. die Prüfung und Genehmigung vor.
\subparagraph{} \ref{subpar:pruefunggenehmigung} gilt nicht für den Mailverkehr.

\paragraph{Haftung}
\subparagraph{} Die Nutzer haften bingo e.V. und Dritten für missbräuchliche und
rechtswidrige Nutzung und schuldhaft verursachte Schäden am Bürgernetz
und dem daraus entstehenden Schaden für Dritte in vollem Umfang.

\paragraph{Sonderregelungen}
\subparagraph{} Ausnahmen zu den Nutzungsbestimmungen bedürfen der schriftlichen
Bestätigung durch den Verein.
\subparagraph{} Eine Abweichung der Schriftformerfordernis kann ebenfalls nur schriftlich
erfolgen.

\paragraph{Änderungsvorbehalt}
\subparagraph{} bingo e.V. behält sich das Recht vor, die Nutzungsbestimmungen jederzeit zu
ändern oder zu ergänzen.

\paragraph{Unwirksamkeit}
\subparagraph{} Sollte eine einzelne Bestimmung dieser Nutzungsbestimmungen unwirksam
sein oder werden, so berührt dies die restlichen Bestimmungen nicht.
\subparagraph{} Es gilt eine dem Zweck der Bestimmung entsprechende oder nahe
kommende Regelung, mit der der Nutzer sich einverstanden gezeigt haben
müsste, wenn er die Unwirksamkeit der eigentlichen Bestimmung gekannt
hätte.
\subparagraph{} Eine entsprechende Regelung gilt für Bestimmungen, die eventuell
unvollständig sind.

\end{multicols}

\vfill
--- \emph{Bürgernetz Ingolstadt -- bingo e.V., Ingolstadt, 01.04.2000}
