\section{Beitragsordnung bingo e.V.} \label{sec:beitragsordnung}
\rowcolors{1}{}{lightgray}
\begin{tabu}{X[3]XX[r]X}
	\rowfont{\bfseries}
	Typ                                                             & Beitrags\-klasse& Beitrag/Jahr      & Postfächer \\
	Natürliche Einzelpersonen                                       & K10             & \EUR{30,-}        & max. 1 \\
	Familien (bis zu 2 Erw. + Kinder bis max. 18 Jahre) Mitbenutzer & K20             & \EUR{50,-}        & max. 10\\
	Vereine (gemeinnützig) Mitbenutzer                              & K30             & \EUR{60,-}        & max. 10\\
	Schulen Mitbenutzer                                             & K40             & \EUR{60,-}        & max. 10\\
	Natürliche Fördermitglieder																			& K42             & \EUR{120,-}       & max. 1 \\
	Vereine (nicht gemeinnützig) Mitbenutzer                        & K50             & \EUR{120,-}       & max. 15\\
	Firmen bis 10 Mitarbeiter, Vereine Mitbenutzer                  & K60             & \EUR{120,-}       & max. 10\\
	Firmen bis 50 Mitarbeiter Mitbenutzer                           & K70             & \EUR{180,-}       & max. 20\\
	Firmen ab 50 Mitarbeiter Mitbenutzer                            & K80             & \EUR{360,-}       & max. 80\\
	sonstige Institutionen Mitbenutzer    					                & K90             &										&\\
\end{tabu}

Der Beitrag ist steuerbegünstigt und für das jeweilige Jahr im Voraus zu entrichten.
Von jedem neuen Mitglied ist eine Aufnahmegebühr von \EUR{10,-} zu entrichten.
Jeder Benutzer hat die Möglichkeit, eine Homepage anzulegen
und bekommt ein individuelles Postfach mit vorgeschaltetem Spam- und Virenfilter zugeteilt.
Es können beliebig viele Aliase angelegt werden, sofern noch verfügbar.

Mitbenutzer haben auch die Möglichkeit, unsere Seminare zum Mitgliedspreis zu besuchen.
Nur der Hauptbenutzer ist Mitglied im Sinne der Satzung und ist entsprechend stimmberechtigt.
Für eine Familienmitgliedschaft ist eine Familienzugehörigkeit nachzuweisen.

Bei Wechsel der Beitragsklasse ist eine Gebühr von \EUR{5,-}  zu entrichten.
Vor dem Wechsel gezahlte Beiträge werden auf den neu zu berechnenden Betrag angerechnet.
Eine Rückerstattung rechnerisch zuviel bezahlter Beiträge erfolgt nicht.
Wird die Mitgliedschaft nicht bis 30.11. zum Jahresende gekündigt,
verlängert sie sich automatisch um ein weiteres Jahr.

Die Nutzungsberechtigung des T-D1-Company Vertrages erlischt mit dem Austritt aus dem Verein.

\hfill --- \textit{Bürgernetz Ingolstadt -- bingo e.V., Ingolstadt, 06.06.2005}
\newpage
